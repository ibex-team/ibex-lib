\chapter{Modeling your Problem}\label{chap:mod}

\section{What we mean by ``variable'' and ``function''}

Before starting, let us rule out a potential ambiguity.

The purpose of this chapter is basically to show how to declare {\it variables} and {\it functions}.
But, since we are in the C++ programming language, the term {\it variable} and
{\it function} already refers to something precise. For instance, the following
piece of code introduces a {\it function} \cf{sum} and a {\it variable} \cf{x}:
\begin{lstlisting}
int sum(int x, int y) { 
  return x+y;
}

int x=2;
\end{lstlisting}
The variable $x$ may represent, say, the balance of a bank account.
The account number is what we call the {\it semantic} of $x$, that is, what $x$ is supposed
to represent in the user's mind. So, on one side,
we have {\it what we write}, that is, a program with variables and functions,
 and on the other side, {\it what we represent}, that is, concepts 
like a bank account.

With \ibex, we write programs to represent mathematical concepts
that are also called {\it variables} and {\it functions}.
The mapping $(x,y)\mapsto \sin(x+y)$ is an example of function that
we want to represent. It shall not be confused with the function \cf{sum}
above.

To avoid ambiguity, we shall talk about {\it mathematical}
variables (resp. functions) versus {\it program} variables (resp. functions).
We will also use italic symbol like $x$ to denote a mathematical variable
and postscript symbols like \cf{x} for program variables.
In most of our discussions, variables and functions will refer
to the mathematical objects so that the mathematical meaning will be often the implicit one. 

\section{Creating variables}

Mathematical variables are represented by objects of the class \hcf{Variable}.

\subsection{Default construction}

The following piece of code creates a variable \cf{x} and prints it.

\begin{lstlisting}
  Variable x;
  cout << x << endl;
\end{lstlisting}

The first instruction creates a (program) variable \cf{x}. It is initialized by default, since
no argument are given here to the constructor.
By default, the variable is real (or {\it scalar}), meaning it is not a vector nor a matrix. 
Furthermore, the (mathematical) variable has a name that is automatically
generated. Of course, the name of the mathematical variable does not necessarily correspond to the name of the 
program variable.
For instance, \cf{x} is the name of a C++ variable but the corresponding 
mathematical variable is named {\it \_x\_0}.
The second instruction prints the name of the mathematical variable on the standard output:

\begin{lstlisting}
_x_0
\end{lstlisting}

It is possible to rename variables, see \S\ref{sec:mod-var-name}.

\subsection{Vector and matrix variables}\label{sec:mod-var-vec}

Like in Matlab, Variables can be vectors or matrices. To create a $n$-dimensional vector variable, just
give the number $n$ as an arguement to the constructor:

\begin{lstlisting}
  Variable y(3);   // creates a 3-dimensional vector
\end{lstlisting}

To create a $m\times n$ matrix, give $m$ (number of rows) and $n$ (number of columns) as arguments:

\begin{lstlisting}
  Variable z(2,3);   // creates a 2*3-dimensional matrix
\end{lstlisting}

We can go like this up to 3 dimensional arrays:

\begin{lstlisting}
  Variable t(2,3,4);   // creates a 2*3*4-dimensional array
\end{lstlisting}


\subsection{Renaming variables}\label{sec:mod-var-name}
Usually, you don't really care about the names of mathematical variables since you handle
program variables in your code.
However, if you want a more user-friendly display, you can specify
the name of the variable as a last argument to the constructor.

In the following example, we create a scalar, a vector and a matrix variable each
time with a chosen name.

\begin{lstlisting}
  Variable x("x");   // creates a real variable named "x"
  Variable y(3,"y");   // creates a vector variable named "y"
  Variable z(2,3,"z");   // creates a matrix variable named "z"
  cout << x << " " << y << " " << z << endl;
\end{lstlisting}

Now, the display is:
\begin{lstlisting}
x y z
\end{lstlisting}


\section{Creating functions}

Mathematical functions are represented by objects of the class \hcf{Function}.
The following piece of code creates the function
$(x,y)\mapsto x+y$:

\begin{lstlisting}	
  Variable x("x");
  Variable y("y");
  Function f(x,y,sin(x+y));
  cout << f << endl;
\end{lstlisting}

The display is:
\begin{lstlisting}
_f_0:(x,y)->sin((x+y))
\end{lstlisting}
%_f_0:(_x_0,_x_1)->sin((_x_0+_x_1))
\section{Evaluation}

We show now how to calculate the image of a box by a function $f$ or, in short, how
to perform an {\it interval evaluation}.

We assume that the function \cf{f} has been created to represent $(x,y)\mapsto \sin(x+y)$ (see the previous paragraph).

We start by building a box (as explained in Chapter \ref{chap:arith}).
The box must have as many components as the function has arguments, here, 2.

Then we simply call \hcf{f.eval(...)} to get the image of the box by \cf{f}:

\begin{lstlisting}
  double _box[][2]= {{1,2},{3,4}};
  IntervalVector box(2,_box);         // create the box ([1,2];[3,4])

  cout << "initial box=" << box << endl;
  cout << "f(box)=" << f.eval(box) << endl; 
\end{lstlisting}



